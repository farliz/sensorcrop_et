\input farliz_garamondpro

\setupinteraction[state=start,color=mgreen,contrastcolor=]

%\definelayer[bground][x=0mm, y=0mm, width=\paperwidth, height=\paperheight] 
%\setupbackgrounds[page][background={bground}]

\setupheadertexts[i3lab][\#Innovación Yubox]
\usebtxdataset[ref]
\usebtxdefinitions[apa]
\setupbtx[apa:cite]
[etallimit=3, etaldisplay=1]


\starttext

\setlayer[bground]{\externalfigure[bg1]}
\blackrule[height=2pt, color=mgreen, width=\textwidth]
\blank[big]
\bTABLE
\bTR\bTD[width=14cm] \externalfigure[sc][width=7cm]\eTD \bTD {\bf Equipo Técnico}  \eTD\eTR
\bTR\bTD \externalfigure[yubox][width=6cm]\eTD \bTD  Gabriel Morejón\\Wendy Obregón\\Lizardo Reyna\\Joffre Moreira  \eTD\eTR
\eTABLE
%\blank[1cm]

\blackrule[height=2pt, color=mgreen, width=\textwidth]
\blank[big]
\startcolumns

\subject{Evapotranspiración y el Riego \thinrule}

La {\bf evaporación} en el suelo es el proceso mediante el cual el agua líquida se convierte en vapor de agua y se elimina en forma de vapor hacia la atmósfera. 

\blank[big]

La {\bf transpiración} consiste en la vaporización del agua líquida contenida en los tejidos vegetales y la eliminación del vapor a la atmósfera. Los cultivos pierden predominantemente su agua a través de los estomas. Estas son pequeñas aberturas en la hoja de la planta a través de las cuales pasan los gases y el vapor de agua \cite[authoryear][fao,PEREIRA20154]. 

\blank[big]

La suma de estas dos cantiades de agua eliminadas hacia la atmósfera se denomina {\bf evapotranspiración} (ET), se mide en mm dia\high{-1} y es un dato clave para la programación del riego en un cultivo (Figura 1).


\blank[big]
\startframedtext[width=\textwidth, framecorner=round]
  {\bf Problema:}\\
En nuestro contexto agrícola, son escasos los agricultores y técnicos que utilizan este parámetro para reponer la cantidad de agua necesaria para el cultivo (regar). En su lugar, la práctica del {\em riego} se limita a proveer de agua a un cultivo sin conocimiento previo de las neceisades reales, resultando en la mayoría de los casos, una {\bf aplicación excesiva} de agua en el suelo.
\stopframedtext
\placefigure[][]{Evapotranspiracion}{\externalfigure[et][width=.9\textwidth]}

%\subject{Estimación de la ET}

Se estima la evapotranspiración de referencia (ETo) utilizando un {\em cultivo} de referencia, que usualmente es el cesped. El valor de ETo, es luego modificado por un factor de ajuste llamado coeficiente de cultivo o {\bf Kc}, debido a que no todos los cultivos o plantas transpiran la misma cantidad de agua. El valor resultante se denomina evapotranpiración real o (ETcrop), que es {\bf la cantidad de agua que se debe reponer al cultivo}. En consecuencia, {\bf si llueve}, parte o toda la necesidad de agua del cultivo se repone naturalmente, de lo contrario es necesario programar un riego. 

\subject{Objetivo \thinrule}

Implementar un sistema automatizado y autónomo para la estimación de la ETcrop en un cultivo agrícola para determinar las necesidades de agua.

\subject{Metodología \thinrule}

Aquí va el esquema y descripción de nuestra solución con Yubox y AGRO


\stopcolumns

\placefigure[][water]{Esquema del contenido de agua en el
  suelo. {\bf Saturación}. Es el exceso de agua en el suelo, los poros
del suelo están completamente ocupados por agua sin dejar espacio para
el intercambio de gases y otras funciones importantes para el
desarrollo de los cultivos. {\bf Capacidad de campo}. Es la cantidad
óptima de agua en el suelo donde los cultivos pueden desarrollarse y
producir con mayor eficiencia. {\bf Punto de marchitez
  permanente}. Sucede cuando el nivel de agua en el suelo ha sido muy
baja por un periodo prologado que la planta yo no es capaz de
recuperarse}{\externalfigure [water][width=.9\textwidth]}

\startcolumns
\subject{Solución y Alcance \thinrule}

El agua en el suelo debe permitir varios procesos entre ellos, la
disolución y transporte de nutrientes a la planta, el intercambio de
gases, regular la temperatura y crear condiciones para la actividad
microbiana. Sin embargo, el déficit o el exceso de agua en el suelo
conlleva a afectaciones importantes en el desarrollo y productividad
de los cultivos (\in{Fig.}[water]).

\blank[big]

El exceso de agua en el suelo provoca un tipo de estrés secundario
llamado {\bf hipoxia} que afecta su capacidad para aportar O\low{2}
a las raíces afectando su crecimiento y provoca cambios fisiológicos
en las plantas. Además el exceso de agua promueve el crecimiento de
hongos y ataque de plagas en los cultivos.

\blank[big]

El estrés hídrico sucede cuando la demanda de agua es mayor a la
disponible en el suelo en un periodo de tiempo. Este tipo de estrés
tiene afectaciones directas al cultivo como reducción del crecimiento,
cierre de los estomas de las hojas, la actividad fotosintética
disminuye o se detiene, falta de nutrientes.

\blank[big]

Estos tipos de estrés tienen una incidencia negativa en el desarrollo
y producción  de los cultivos, ocasionando pérdidas económicas
importantes al agricultor.



\startframedtext[width=\textwidth, framecorner=round,
  foregroundcolor=white, background=color, backgroundcolor=mgreen,
  style=\ss, frame=off]
{\bf Solución:}\\

\# SENSOR{\bf CROP} es la alternativa que soluciona el problema del
agua en el suelo. Este dispositivo estima de forma automática y
autónoma, la cantidad de agua que hay que reponer en el suelo para
suministrarla en un riego programado. Así controlaremos técnicamente
el riego de los cultivos garantizando un óptima producción.
\stopframedtext

\startframedtext[width=\textwidth, framecorner=round]
{\bf Impacto:}\\
  Con el manejo apropiado del riego, se evitan los efectos negativos en
  los cultivos por exceso o falta de agua, por el contrario, proporciona las
  condiciones óptimas para el desarrollo de los cultivos, lo que
  resulta en una buena producción y conservación del suelo para las
  futuras siembras.
  
\stopframedtext



\subject{Referencias}
\switchtobodyfont[10pt]
\placelistofpublications
  [criterium=all]
\stopcolumns


\stoptext
